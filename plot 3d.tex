
\begin{eulercomment}
\eulerheading{Menggambar Plot 3D dengan EMT}
\begin{eulercomment}
Ini adalah pengantar untuk plot 3D dalam Euler. Kita memerlukan plot
3D untuk memvisualisasikan sebuah fungsi dari dua variabel.

Euler menggambar fungsi-fungsi seperti itu menggunakan algoritma
pengurutan untuk menyembunyikan bagian-bagian di latar belakang.
Secara umum, Euler menggunakan proyeksi pusat. Defaultnya adalah dari
kuadran positif x-y menuju titik asal x=y=z=0, tetapi sudut=0°
menghadap ke arah sumbu y. Sudut pandangan dan tinggi dapat diubah.

Euler dapat membuat plot:

- permukaan dengan bayangan dan garis level atau rentang level,\\
- awan titik,\\
- kurva parametrik,\\
- permukaan implisit.

Plot 3D dari sebuah fungsi menggunakan plot3d. Cara termudah adalah
dengan memplot ekspresi dalam x dan y. Parameter r mengatur rentang
plot di sekitar (0,0).
\end{eulercomment}
\begin{eulerprompt}
>aspect(1.5); plot3d("x^2+sin(y)",-5,5,0,6*pi):
\end{eulerprompt}
\eulerimg{17}{images/plot 3d-001.png}
\begin{eulerprompt}
> 
>plot3d("x^2+x*sin(y)",-5,5,0,6*pi):
\end{eulerprompt}
\eulerimg{17}{images/plot 3d-002.png}
\begin{eulercomment}
Silakan lakukan modifikasi agar gambar "talang bergelombang" tersebut tidak lurus melainkan melengkung/melingkar, baik
melingkar secara mendatar maupun melingkar turun/naik (seperti papan peluncur pada kolam renang. Temukan rumusnya.
\end{eulercomment}
\eulerheading{Fungsi Dua Variabel}
\begin{eulercomment}
Untuk grafik dari sebuah fungsi, gunakan

- ekspresi sederhana dalam x dan y,\\
- nama fungsi dua variabel,\\
- atau matriks data.

Pengaturan default adalah tampilan grid berwarna dengan grid yang
berbeda di kedua sisi. Perlu diingat bahwa jumlah interval grid
default adalah 10, tetapi plot menggunakan jumlah default 40x40
persegi untuk membangun permukaan. Ini dapat diubah.

- n=40, n=[40,40]: jumlah garis grid dalam setiap arah\\
- grid=10, grid=[10,10]: jumlah garis grid dalam setiap arah.

Kami menggunakan n=40 dan grid=10 sebagai default.
\end{eulercomment}
\begin{eulerprompt}
>plot3d("x^2+y^2"):
\end{eulerprompt}
\eulerimg{17}{images/plot 3d-003.png}
\begin{eulercomment}
Interaksi pengguna dimungkinkan dengan parameter \textgreater{}user. Pengguna dapat
menekan tombol-tombol berikut.

- kiri, kanan, atas, bawah: mengubah sudut pandang\\
- +, -: memperbesar atau memperkecil\\
- a: menghasilkan anaglif (lihat di bawah)\\
- l: mengaktifkan atau menonaktifkan sumber cahaya (lihat di bawah)\\
- spasi: mengembalikan ke pengaturan awal\\
- enter: mengakhiri interaksi
\end{eulercomment}
\begin{eulerprompt}
>plot3d("exp(-x^2+y^2)",>user, ...
>  title="Turn with the vector keys (press return to finish)"):
\end{eulerprompt}
\eulerimg{17}{images/plot 3d-004.png}
\begin{eulercomment}
Rentang plot untuk fungsi dapat ditentukan dengan

- a, b: rentang x\\
- c, d: rentang y\\
- r: kotak simetris di sekitar (0,0).\\
- n: jumlah subinterval untuk plot.

Ada beberapa parameter untuk menyesuaikan fungsi atau mengubah
tampilan grafik.

fscale: mengubah skala nilai-nilai fungsi (default adalah \textless{}fscale).\\
skala: angka atau vektor 1x2 untuk mengubah skala ke arah x- dan y.\\
frame: jenis bingkai (default 1).
\end{eulercomment}
\begin{eulerprompt}
>plot3d("exp(-(x^2+y^2)/5)",r=10,n=80,fscale=4,scale=1.2,frame=3,>user):
\end{eulerprompt}
\eulerimg{17}{images/plot 3d-005.png}
\begin{eulercomment}
Tampilan dapat diubah dengan berbagai cara.

- jarak: jarak pandangan ke plot.\\
- zoom: nilai zoom.\\
- sudut: sudut terhadap sumbu y negatif dalam radian.\\
- tinggi: tinggi tampilan dalam radian.

Nilai default dapat diperiksa atau diubah dengan fungsi view(). Fungsi
ini mengembalikan parameter dalam urutan di atas.
\end{eulercomment}
\begin{eulerprompt}
>view
\end{eulerprompt}
\begin{euleroutput}
  [5,  2.6,  2,  0.4]
\end{euleroutput}
\begin{eulercomment}
Jarak yang lebih dekat memerlukan zoom yang lebih sedikit. Efeknya
lebih mirip lensa sudut lebar.

Pada contoh berikutnya, sudut=0 dan tinggi=0 dilihat dari sumbu y
negatif. Label sumbu untuk y disembunyikan dalam kasus ini.
\end{eulercomment}
\begin{eulerprompt}
>plot3d("x^2+y",distance=3,zoom=1,angle=pi/2,height=0):
\end{eulerprompt}
\eulerimg{17}{images/plot 3d-006.png}
\begin{eulercomment}
Plot selalu terlihat ke pusat kubus plot. Anda dapat memindahkan
pusatnya dengan parameter pusat.
\end{eulercomment}
\begin{eulerprompt}
>plot3d("x^4+y^2",a=0,b=1,c=-1,d=1,angle=-20°,height=20°, ...
>  center=[0.4,0,0],zoom=5):
\end{eulerprompt}
\eulerimg{17}{images/plot 3d-007.png}
\begin{eulercomment}
Plot tersebut diperkecil agar sesuai dengan kubus satuan saat dilihat.
Jadi, tidak perlu mengubah jarak atau zoom tergantung pada ukuran
plot. Label-labelnya tetap mengacu pada ukuran sebenarnya.

Jika Anda mematikan fitur ini dengan scale=false, Anda perlu
memastikan bahwa plot tetap muat dalam jendela plotting, dengan
mengubah jarak pandang atau zoom, dan menggeser pusatnya.
\end{eulercomment}
\begin{eulerprompt}
>plot3d("5*exp(-x^2-y^2)",r=2,<fscale,<scale,distance=13,height=50°, ...
>  center=[0,0,-2],frame=3):
\end{eulerprompt}
\eulerimg{17}{images/plot 3d-008.png}
\begin{eulercomment}
Grafik polar juga tersedia. Parameter polar=true menggambar grafik
polar. Fungsi tetap harus menjadi fungsi dari x dan y. Parameter
"fscale" mengubah skala fungsi dengan skala sendiri. Jika tidak,
fungsi akan diubah skala agar sesuai dalam sebuah kubus.
\end{eulercomment}
\begin{eulerprompt}
>plot3d("1/(x^2+y^2+1)",r=5,>polar, ...
>fscale=2,>hue,n=100,zoom=4,>contour,color=blue):
\end{eulerprompt}
\eulerimg{17}{images/plot 3d-009.png}
\begin{eulerprompt}
>function f(r) := exp(-r/2)*cos(r); ...
>plot3d("f(x^2+y^2)",>polar,scale=[1,1,0.4],r=pi,frame=3,zoom=4):
\end{eulerprompt}
\eulerimg{17}{images/plot 3d-010.png}
\begin{eulercomment}
Parameter "rotate" memutar suatu fungsi dalam sumbu x sekitar sumbu x.

- rotate=1: Menggunakan sumbu x\\
- rotate=2: Menggunakan sumbu z
\end{eulercomment}
\begin{eulerprompt}
>plot3d("x^2+1",a=-1,b=1,rotate=true,grid=5):
\end{eulerprompt}
\eulerimg{17}{images/plot 3d-011.png}
\begin{eulerprompt}
>plot3d("x^2+1",a=-1,b=1,rotate=2,grid=5):
\end{eulerprompt}
\eulerimg{17}{images/plot 3d-012.png}
\begin{eulerprompt}
>plot3d("sqrt(25-x^2)",a=0,b=5,rotate=1):
\end{eulerprompt}
\eulerimg{17}{images/plot 3d-013.png}
\begin{eulerprompt}
>plot3d("x*sin(x)",a=0,b=6pi,rotate=2):
\end{eulerprompt}
\eulerimg{17}{images/plot 3d-014.png}
\begin{eulercomment}
Berikut adalah sebuah plot dengan tiga fungsi.
\end{eulercomment}
\begin{eulerprompt}
>plot3d("x","x^2+y^2","y",r=2,zoom=3.5,frame=3):
\end{eulerprompt}
\eulerimg{17}{images/plot 3d-015.png}
\eulerheading{Plot Kontur}
\begin{eulercomment}
Untuk plot ini, Euler menambahkan garis-garis kisi. Sebagai gantinya,
kita dapat menggunakan garis level dan satu warna atau spektrum warna.
Euler dapat menggambar tinggi fungsi pada plot dengan shading. Dalam
semua plot 3D, Euler dapat menghasilkan anaglyph merah/cyan.

- \textgreater{}hue: Mengaktifkan shading ringan daripada kawat.\\
- \textgreater{}contour: Melakukan plot garis kontur otomatis pada plot.\\
- level=... (atau levels): Sebuah vektor nilai untuk garis kontur.

Defaultnya adalah level="auto", yang menghitung beberapa garis level
secara otomatis. Seperti yang Anda lihat dalam plot, level sebenarnya
adalah rentang level.

Gaya default dapat diubah. Untuk plot kontur berikutnya, kami
menggunakan kisi yang lebih halus dengan 100x100 titik, mengubah skala
fungsi dan plot, dan menggunakan sudut pandang yang berbeda.
\end{eulercomment}
\begin{eulerprompt}
>plot3d("exp(-x^2-y^2)",r=2,n=100,level="thin", ...
> >contour,>spectral,fscale=1,scale=1.1,angle=45°,height=20°):
\end{eulerprompt}
\eulerimg{17}{images/plot 3d-016.png}
\begin{eulerprompt}
>plot3d("exp(x*y)",angle=100°,>contour,color=green):
\end{eulerprompt}
\eulerimg{17}{images/plot 3d-017.png}
\begin{eulercomment}
Pengaturan bayangan bawaan menggunakan warna abu-abu. Namun, juga
tersedia berbagai macam warna dalam spektrum.

- \textgreater{}spektral: Menggunakan skema spektral bawaan\\
- warna=...: Menggunakan warna khusus atau skema spektral

Untuk plot berikut, kita menggunakan skema spektral bawaan dan
meningkatkan jumlah titik untuk mendapatkan tampilan yang sangat
halus.
\end{eulercomment}
\begin{eulerprompt}
>plot3d("x^2+y^2",>spectral,>contour,n=100):
\end{eulerprompt}
\eulerimg{17}{images/plot 3d-018.png}
\begin{eulercomment}
Daripada garis level otomatis, kita juga dapat mengatur nilai-nilai
dari garis level. Hal ini akan menghasilkan garis level yang tipis
daripada rentang level.
\end{eulercomment}
\begin{eulerprompt}
>plot3d("x^2-y^2",0,5,0,5,level=-1:0.1:1,color=redgreen):
\end{eulerprompt}
\eulerimg{17}{images/plot 3d-019.png}
\begin{eulercomment}
Dalam plot berikut, kita menggunakan dua rentang level yang sangat
luas dari -0.1 hingga 1, dan dari 0.9 hingga 1. Ini dimasukkan sebagai
matriks dengan batas level sebagai kolom.

Selain itu, kita menimbulkan grid dengan 10 interval di setiap arah.
\end{eulercomment}
\begin{eulerprompt}
>plot3d("x^2+y^3",level=[-0.1,0.9;0,1], ...
>  >spectral,angle=30°,grid=10,contourcolor=gray):
\end{eulerprompt}
\eulerimg{17}{images/plot 3d-020.png}
\begin{eulercomment}
Dalam contoh berikut, kita menggambar himpunan, di mana

\end{eulercomment}
\begin{eulerformula}
\[
f(x, y) = x^y - y^x = 0
\]
\end{eulerformula}
\begin{eulercomment}
Kita menggunakan satu garis tipis untuk garis level.
\end{eulercomment}
\begin{eulerprompt}
>plot3d("x^y-y^x",level=0,a=0,b=6,c=0,d=6,contourcolor=red,n=100):
\end{eulerprompt}
\eulerimg{17}{images/plot 3d-021.png}
\begin{eulercomment}
Mungkin untuk menampilkan sebuah bidang kontur di bawah plot. Warna
dan jarak ke plot dapat ditentukan.
\end{eulercomment}
\begin{eulerprompt}
>plot3d("x^2+y^4",>cp,cpcolor=green,cpdelta=0.2):
\end{eulerprompt}
\eulerimg{17}{images/plot 3d-022.png}
\begin{eulercomment}
Berikut beberapa gaya tambahan. Kami selalu mematikan bingkai (frame),
dan menggunakan berbagai skema warna untuk plot dan grid.
\end{eulercomment}
\begin{eulerprompt}
>figure(2,2); ...
>expr="y^3-x^2"; ...
>figure(1);  ...
>  plot3d(expr,<frame,>cp,cpcolor=spectral); ...
>figure(2);  ...
>  plot3d(expr,<frame,>spectral,grid=10,cp=2); ...
>figure(3);  ...
>  plot3d(expr,<frame,>contour,color=gray,nc=5,cp=3,cpcolor=greenred); ...
>figure(4);  ...
>  plot3d(expr,<frame,>hue,grid=10,>transparent,>cp,cpcolor=gray); ...
>figure(0):
\end{eulerprompt}
\eulerimg{17}{images/plot 3d-023.png}
\begin{eulercomment}
Ada beberapa skema spektral lain, yang diberi nomor dari 1 hingga 9.
Tetapi Anda juga dapat menggunakan warna=nilai, di mana nilai

- spektral: untuk rentang dari biru hingga merah\\
- putih: untuk rentang yang lebih samar\\
- kuningbiru, hijauungu, birukuning, merahhijau\\
- birukuning, unguhijau, kuningbiru, hijaumerah
\end{eulercomment}
\begin{eulerprompt}
>figure(3,3); ...
>for i=1:9;  ...
>  figure(i); plot3d("x^2+y^2",spectral=i,>contour,>cp,<frame,zoom=4);  ...
>end; ...
>figure(0):
\end{eulerprompt}
\eulerimg{17}{images/plot 3d-024.png}
\begin{eulercomment}
Sumber cahaya dapat diubah dengan tombol "l" dan tombol kursor selama
interaksi pengguna. Ini juga dapat diatur dengan parameter.

- light: arah cahaya\\
- amb: cahaya ambien antara 0 dan 1

Perhatikan bahwa program ini tidak membedakan sisi plot. Tidak ada
bayangan. Untuk itu, Anda akan memerlukan Povray.
\end{eulercomment}
\begin{eulerprompt}
>plot3d("-x^2-y^2", ...
>  hue=true,light=[0,1,1],amb=0,user=true, ...
>  title="Press l and cursor keys (return to exit)"):
\end{eulerprompt}
\eulerimg{17}{images/plot 3d-025.png}
\begin{eulercomment}
Parameter warna mengubah warna permukaan. Warna dari garis level juga
dapat diubah.
\end{eulercomment}
\begin{eulerprompt}
>plot3d("-x^2-y^2",color=rgb(0.2,0.2,0),hue=true,frame=false, ...
>  zoom=3,contourcolor=red,level=-2:0.1:1,dl=0.01):
\end{eulerprompt}
\eulerimg{17}{images/plot 3d-026.png}
\begin{eulercomment}
Warna 0 memberikan efek pelangi yang istimewa.
\end{eulercomment}
\begin{eulerprompt}
>plot3d("x^2/(x^2+y^2+1)",color=0,hue=true,grid=10):
\end{eulerprompt}
\eulerimg{17}{images/plot 3d-027.png}
\begin{eulercomment}
Permukaannya juga bisa transparan.
\end{eulercomment}
\begin{eulerprompt}
>plot3d("x^2+y^2",>transparent,grid=10,wirecolor=red):
\end{eulerprompt}
\eulerimg{17}{images/plot 3d-028.png}
\eulerheading{Plot Implisit}
\begin{eulercomment}
Ada juga plot implisit dalam tiga dimensi. Euler menghasilkan potongan
melalui objek-objek tersebut. Fitur dari plot3d mencakup plot
implisit. Plot ini menampilkan himpunan nol dari suatu fungsi dalam
tiga variabel.\\
Solusi dari

\end{eulercomment}
\begin{eulerformula}
\[
f(x,y,z) = 0
\]
\end{eulerformula}
\begin{eulercomment}
dapat divisualisasikan dalam potongan sejajar dengan bidang x-y,
bidang x-z, dan bidang y-z.

- implicit=1: potongan sejajar dengan bidang y-z\\
- implicit=2: potongan sejajar dengan bidang x-z\\
- implicit=4: potongan sejajar dengan bidang x-y

Tambahkan nilai-nilai ini, jika Anda suka. Dalam contoh ini, kita
memplot

\end{eulercomment}
\begin{eulerformula}
\[
M = \{ (x,y,z) : x^2+y^3+zy=1 \}
\]
\end{eulerformula}
\begin{eulerprompt}
>plot3d("x^2+y^3+z*y-1",r=5,implicit=3):
\end{eulerprompt}
\eulerimg{17}{images/plot 3d-029.png}
\begin{eulerprompt}
>c=1; d=1;
>plot3d("((x^2+y^2-c^2)^2+(z^2-1)^2)*((y^2+z^2-c^2)^2+(x^2-1)^2)*((z^2+x^2-c^2)^2+(y^2-1)^2)-d",r=2,<frame,>implicit,>user): 
\end{eulerprompt}
\eulerimg{17}{images/plot 3d-030.png}
\begin{eulerprompt}
>plot3d("x^2+y^2+4*x*z+z^3",>implicit,r=2,zoom=2.5):
\end{eulerprompt}
\eulerimg{17}{images/plot 3d-031.png}
\eulerheading{Plotting Data 3D}
\begin{eulercomment}
Sama seperti plot2d, plot3d menerima data. Untuk objek 3D, Anda perlu
menyediakan matriks nilai x-, y-, dan z-, atau tiga fungsi atau
ekspresi fx(x,y), fy(x,y), fz(x,y).

\end{eulercomment}
\begin{eulerformula}
\[
\gamma(t,s) = (x(t,s),y(t,s),z(t,s))
\]
\end{eulerformula}
\begin{eulercomment}
Karena x, y, z adalah matriks, kami mengasumsikan bahwa (t,s) berjalan
melalui grid persegi. Sebagai hasilnya, Anda dapat membuat
gambar-gambar segi empat di dalam ruang.

Anda dapat menggunakan bahasa matriks Euler untuk menghasilkan
koordinat dengan efektif.

Pada contoh berikut, kami menggunakan vektor nilai t dan vektor kolom
nilai s untuk memparametrisasi permukaan bola. Dalam gambaran ini,
kami dapat menandai wilayah-wilayah, dalam kasus kami adalah wilayah
kutub.
\end{eulercomment}
\begin{eulerprompt}
>t=linspace(0,2pi,180); s=linspace(-pi/2,pi/2,90)'; ...
>x=cos(s)*cos(t); y=cos(s)*sin(t); z=sin(s); ...
>plot3d(x,y,z,>hue, ...
>color=blue,<frame,grid=[10,20], ...
>values=s,contourcolor=red,level=[90°-24°;90°-22°], ...
>scale=1.4,height=50°):
\end{eulerprompt}
\eulerimg{17}{images/plot 3d-032.png}
\begin{eulercomment}
Berikut ini adalah contoh, yang merupakan grafik dari sebuah fungsi.
\end{eulercomment}
\begin{eulerprompt}
>t=-1:0.1:1; s=(-1:0.1:1)'; plot3d(t,s,t*s,grid=10):
\end{eulerprompt}
\eulerimg{17}{images/plot 3d-033.png}
\begin{eulercomment}
Namun, kita dapat membuat berbagai jenis permukaan. Berikut ini adalah
permukaan yang sama dalam bentuk fungsi

\end{eulercomment}
\begin{eulerformula}
\[
x = y \, z
\]
\end{eulerformula}
\begin{eulerprompt}
>plot3d(t*s,t,s,angle=180°,grid=10):
\end{eulerprompt}
\eulerimg{17}{images/plot 3d-034.png}
\begin{eulercomment}
Dengan lebih banyak usaha, kita bisa menghasilkan banyak permukaan.

Pada contoh berikut, kita membuat tampilan berbayang dari bola yang
terdistorsi. Koordinat biasa untuk bola tersebut adalah

\end{eulercomment}
\begin{eulerformula}
\[
\gamma(t,s) = (\cos(t)\cos(s),\sin(t)\sin(s),\cos(s))
\]
\end{eulerformula}
\begin{eulercomment}
dengan

\end{eulercomment}
\begin{eulerformula}
\[
0 \le t \le 2\pi, \quad \frac{-\pi}{2} \le s \le \frac{\pi}{2}.
\]
\end{eulerformula}
\begin{eulercomment}
Kita merubahnya dengan faktor

\end{eulercomment}
\begin{eulerformula}
\[
d(t,s) = \frac{\cos(4t)+\cos(8s)}{4}.
\]
\end{eulerformula}
\begin{eulerprompt}
>t=linspace(0,2pi,320); s=linspace(-pi/2,pi/2,160)'; ...
>d=1+0.2*(cos(4*t)+cos(8*s)); ...
>plot3d(cos(t)*cos(s)*d,sin(t)*cos(s)*d,sin(s)*d,hue=1, ...
>  light=[1,0,1],frame=0,zoom=5):
\end{eulerprompt}
\eulerimg{17}{images/plot 3d-035.png}
\begin{eulercomment}
Tentu saja, awan titik juga memungkinkan. Untuk memplot data titik di
dalam ruang, kita memerlukan tiga vektor untuk koordinat titik-titik
tersebut.

Gaya-gaya ini sama seperti dalam plot2d dengan points=true;
\end{eulercomment}
\begin{eulerprompt}
>n=500;  ...
>  plot3d(normal(1,n),normal(1,n),normal(1,n),points=true,style="."):
\end{eulerprompt}
\eulerimg{17}{images/plot 3d-036.png}
\begin{eulercomment}
Hal ini juga mungkin untuk menggambar kurva dalam 3D. Dalam hal ini,
lebih mudah untuk menghitung titik-titik kurva terlebih dahulu. Untuk
kurva dalam bidang, kami menggunakan urutan koordinat dan parameter
kawat=true.
\end{eulercomment}
\begin{eulerprompt}
>t=linspace(0,8pi,500); ...
>plot3d(sin(t),cos(t),t/10,>wire,zoom=3):
\end{eulerprompt}
\eulerimg{17}{images/plot 3d-037.png}
\begin{eulerprompt}
>t=linspace(0,4pi,1000); plot3d(cos(t),sin(t),t/2pi,>wire, ...
>linewidth=3,wirecolor=blue):
\end{eulerprompt}
\eulerimg{17}{images/plot 3d-038.png}
\begin{eulerprompt}
>X=cumsum(normal(3,100)); ...
> plot3d(X[1],X[2],X[3],>anaglyph,>wire):
\end{eulerprompt}
\eulerimg{17}{images/plot 3d-039.png}
\begin{eulercomment}
EMT juga dapat membuat plot dalam mode anaglif. Untuk melihat plot
seperti itu, Anda memerlukan kacamata merah/cyan.
\end{eulercomment}
\begin{eulerprompt}
> plot3d("x^2+y^3",>anaglyph,>contour,angle=30°):
\end{eulerprompt}
\eulerimg{17}{images/plot 3d-040.png}
\begin{eulercomment}
Seringkali, skema warna spektral digunakan untuk plot. Ini menekankan
tinggi dari fungsi tersebut.
\end{eulercomment}
\begin{eulerprompt}
>plot3d("x^2*y^3-y",>spectral,>contour,zoom=3.2):
\end{eulerprompt}
\eulerimg{17}{images/plot 3d-041.png}
\begin{eulercomment}
Euler juga dapat menggambar permukaan yang diberparameterisasi, ketika
parameter-parameter tersebut adalah nilai-nilai x, y, dan z dari
sebuah gambar grid persegi di dalam ruang.

Untuk demo berikutnya, kami menyiapkan parameter-parameter u dan v,
dan menghasilkan koordinat ruang dari parameter-parameter ini.
\end{eulercomment}
\begin{eulerprompt}
>u=linspace(-1,1,10); v=linspace(0,2*pi,50)'; ...
>X=(3+u*cos(v/2))*cos(v); Y=(3+u*cos(v/2))*sin(v); Z=u*sin(v/2); ...
>plot3d(X,Y,Z,>anaglyph,<frame,>wire,scale=2.3):
\end{eulerprompt}
\eulerimg{17}{images/plot 3d-042.png}
\begin{eulercomment}
Berikut adalah contoh yang lebih rumit, yang sangat megah dengan
kacamata merah/cyan.
\end{eulercomment}
\begin{eulerprompt}
>u:=linspace(-pi,pi,160); v:=linspace(-pi,pi,400)';  ...
>x:=(4*(1+.25*sin(3*v))+cos(u))*cos(2*v); ...
>y:=(4*(1+.25*sin(3*v))+cos(u))*sin(2*v); ...
> z=sin(u)+2*cos(3*v); ...
>plot3d(x,y,z,frame=0,scale=1.5,hue=1,light=[1,0,-1],zoom=2.8,>anaglyph):
\end{eulerprompt}
\eulerimg{17}{images/plot 3d-043.png}
\eulerheading{Plot Statistik}
\begin{eulercomment}
Diagram batang juga dimungkinkan. Untuk ini, kita harus menyediakan:

- x: vektor baris dengan n+1 elemen\\
- y: vektor kolom dengan n+1 elemen\\
- z: matriks nxn dari nilai-nilai.

z bisa lebih besar, tetapi hanya nilai nxn yang akan digunakan.

Dalam contoh ini, kita pertama-tama menghitung nilai-nilai tersebut.
Kemudian kita menyesuaikan x dan y, sehingga vektor-vektor tersebut
berpusat pada nilai-nilai yang digunakan.
\end{eulercomment}
\begin{eulerprompt}
>x=-1:0.1:1; y=x'; z=x^2+y^2; ...
>xa=(x|1.1)-0.05; ya=(y_1.1)-0.05; ...
>plot3d(xa,ya,z,bar=true):
\end{eulerprompt}
\eulerimg{17}{images/plot 3d-044.png}
\begin{eulercomment}
Sangat mungkin untuk membagi plot permukaan menjadi dua atau lebih
bagian.
\end{eulercomment}
\begin{eulerprompt}
>x=-1:0.1:1; y=x'; z=x+y; d=zeros(size(x)); ...
>plot3d(x,y,z,disconnect=2:2:20):
\end{eulerprompt}
\eulerimg{17}{images/plot 3d-045.png}
\begin{eulercomment}
Jika Anda memuat atau menghasilkan matriks data M dari sebuah file dan
perlu menggambarkannya dalam tiga dimensi, Anda dapat mengubah skala
matriks tersebut menjadi [-1,1] dengan menggunakan fungsi scale(M),
atau Anda juga dapat mengubah skala matriks tersebut dengan
menggunakan \textgreater{}zscale. Ini dapat digabungkan dengan faktor-faktor skala
individual yang diterapkan tambahan.
\end{eulercomment}
\begin{eulerprompt}
>i=1:20; j=i'; ...
>plot3d(i*j^2+100*normal(20,20),>zscale,scale=[1,1,1.5],angle=-40°,zoom=1.8):
\end{eulerprompt}
\eulerimg{17}{images/plot 3d-046.png}
\begin{eulerprompt}
>Z=intrandom(5,100,6); v=zeros(5,6); ...
>loop 1 to 5; v[#]=getmultiplicities(1:6,Z[#]); end; ...
>columnsplot3d(v',scols=1:5,ccols=[1:5]):
\end{eulerprompt}
\eulerimg{17}{images/plot 3d-047.png}
\eulerheading{Permukaan Benda Putar}
\begin{eulerprompt}
>plot2d("(x^2+y^2-1)^3-x^2*y^3",r=1.3, ...
>style="#",color=red,<outline, ...
>level=[-2;0],n=100):
\end{eulerprompt}
\eulerimg{17}{images/plot 3d-048.png}
\begin{eulerprompt}
>ekspresi &= (x^2+y^2-1)^3-x^2*y^3; $ekspresi
\end{eulerprompt}
\begin{eulerformula}
\[
\left(y^2+x^2-1\right)^3-x^2\,y^3
\]
\end{eulerformula}
\begin{eulercomment}
Kami ingin memutar kurva hati sekitar sumbu y. Berikut adalah ekspresi
yang mendefinisikan hati:

\end{eulercomment}
\begin{eulerformula}
\[
f(x,y)=(x^2+y^2-1)^3-x^2.y^3.
\]
\end{eulerformula}
\begin{eulercomment}
Selanjutnya, kami mengatur

\end{eulercomment}
\begin{eulerformula}
\[
x=r.cos(a),\quad y=r.sin(a).
\]
\end{eulerformula}
\begin{eulerprompt}
>function fr(r,a) &= ekspresi with [x=r*cos(a),y=r*sin(a)] | trigreduce; $fr(r,a)
\end{eulerprompt}
\begin{eulerformula}
\[
\left(r^2-1\right)^3+\frac{\left(\sin \left(5\,a\right)-\sin \left(
 3\,a\right)-2\,\sin a\right)\,r^5}{16}
\]
\end{eulerformula}
\begin{eulercomment}
Ini memungkinkan untuk mendefinisikan fungsi numerik, yang
menyelesaikan untuk r, jika a diberikan. Dengan fungsi itu, kita dapat
menggambar hati yang terbalik sebagai permukaan parametrik.
\end{eulercomment}
\begin{eulerprompt}
>function map f(a) := bisect("fr",0,2;a); ...
>t=linspace(-pi/2,pi/2,100); r=f(t);  ...
>s=linspace(pi,2pi,100)'; ...
>plot3d(r*cos(t)*sin(s),r*cos(t)*cos(s),r*sin(t), ...
>>hue,<frame,color=red,zoom=4,amb=0,max=0.7,grid=12,height=50°):
\end{eulerprompt}
\eulerimg{17}{images/plot 3d-051.png}
\begin{eulercomment}
Berikut adalah plot 3D dari gambar di atas yang diputar sekitar sumbu
z. Kami mendefinisikan fungsi, yang menggambarkan objek tersebut.
\end{eulercomment}
\begin{eulerprompt}
>function f(x,y,z) ...
\end{eulerprompt}
\begin{eulerudf}
  r=x^2+y^2;
  return (r+z^2-1)^3-r*z^3;
   endfunction
\end{eulerudf}
\begin{eulerprompt}
>plot3d("f(x,y,z)", ...
>xmin=0,xmax=1.2,ymin=-1.2,ymax=1.2,zmin=-1.2,zmax=1.4, ...
>implicit=1,angle=-30°,zoom=2.5,n=[10,100,60],>anaglyph):
\end{eulerprompt}
\eulerimg{17}{images/plot 3d-052.png}
\eulerheading{Plot 3D Khusus}
\begin{eulercomment}
Fungsi plot3d bagus untuk dimiliki, tetapi tidak memenuhi semua
kebutuhan. Selain rutinitas yang lebih dasar, Anda dapat membuat plot
berbingkai dari objek apa pun yang Anda sukai.

Meskipun Euler bukan program 3D, ini dapat menggabungkan beberapa
objek dasar. Kami mencoba untuk memvisualisasikan sebuah paraboloid
dan tangennya.
\end{eulercomment}
\begin{eulerprompt}
>function myplot ...
\end{eulerprompt}
\begin{eulerudf}
    y=-1:0.01:1; x=(-1:0.01:1)';
    plot3d(x,y,0.2*(x-0.1)/2,<scale,<frame,>hue, ..
      hues=0.5,>contour,color=orange);
    h=holding(1);
    plot3d(x,y,(x^2+y^2)/2,<scale,<frame,>contour,>hue);
    holding(h);
  endfunction
\end{eulerudf}
\begin{eulercomment}
Sekarang, framedplot() menyediakan bingkai-bingkai, dan mengatur
tampilan.
\end{eulercomment}
\begin{eulerprompt}
>framedplot("myplot",[-1,1,-1,1,0,1],height=0,angle=-30°, ...
>  center=[0,0,-0.7],zoom=3):
\end{eulerprompt}
\eulerimg{17}{images/plot 3d-053.png}
\begin{eulercomment}
Dengan cara yang sama, Anda dapat menggambar bidang kontur secara
manual. Perhatikan bahwa plot3d() secara default mengatur jendela ke
fullwindow(), tetapi plotcontourplane() mengasumsikan hal tersebut.
\end{eulercomment}
\begin{eulerprompt}
>x=-1:0.02:1.1; y=x'; z=x^2-y^4;
>function myplot (x,y,z) ...
\end{eulerprompt}
\begin{eulerudf}
    zoom(2);
    wi=fullwindow();
    plotcontourplane(x,y,z,level="auto",<scale);
    plot3d(x,y,z,>hue,<scale,>add,color=white,level="thin");
    window(wi);
    reset();
  endfunction
\end{eulerudf}
\begin{eulerprompt}
>myplot(x,y,z):
\end{eulerprompt}
\eulerimg{27}{images/plot 3d-054.png}
\eulerheading{Animasi}
\begin{eulercomment}
Euler dapat menggunakan bingkai (frames) untuk menghitung animasi
sebelumnya.

Salah satu fungsi yang menggunakan teknik ini adalah fungsi rotate.
Fungsi ini dapat mengubah sudut pandangan dan menggambar ulang plot
3D. Fungsi ini memanggil addpage() untuk setiap plot baru. Akhirnya,
itu menganimasikan plot-plot tersebut.

Silahkan pelajari sumber dari fungsi rotate untuk melihat lebih banyak
detailnya.
\end{eulercomment}
\begin{eulerprompt}
>function testplot () := plot3d("x^2+y^3"); ...
>  rotate("testplot"); testplot():
\end{eulerprompt}
\eulerimg{27}{images/plot 3d-055.png}
\eulerheading{Menggambar Povray}
\begin{eulercomment}
Dengan bantuan berkas Euler povray.e, Euler dapat menghasilkan berkas
Povray. Hasilnya sangat bagus untuk dilihat.

Anda perlu menginstal Povray (32bit atau 64bit) dari
http://www.povray.org/, dan letakkan sub-direktori "bin" dari Povray ke dalam path lingkungan, atau atur variabel "defaultpovray" dengan path lengkap menuju "pvengine.exe".

Antarmuka Povray dari Euler menghasilkan berkas Povray di direktori
rumah pengguna, dan memanggil Povray untuk menguraikan berkas-berkas
ini. Nama berkas default adalah current.pov, dan direktori default
adalah eulerhome(), biasanya c:\textbackslash{}Users\textbackslash{}Username\textbackslash{}Euler. Povray
menghasilkan berkas PNG, yang dapat dimuat oleh Euler ke dalam
notebook. Untuk membersihkan berkas-berkas ini, gunakan povclear().

Fungsi pov3d berada dalam semangat yang sama seperti plot3d. Ini dapat
menghasilkan grafik dari fungsi f(x, y), atau permukaan dengan
koordinat X, Y, Z dalam matriks, termasuk garis level opsional. Fungsi
ini secara otomatis memulai raytracer, dan memuat adegan ke dalam
notebook Euler.

Selain pov3d(), ada banyak fungsi lain yang menghasilkan objek Povray.
Fungsi-fungsi ini mengembalikan string-string yang berisi kode Povray
untuk objek-objek tersebut. Untuk menggunakan fungsi-fungsi ini, mulai
berkas Povray dengan povstart(). Kemudian gunakan writeln(...) untuk
menulis objek-objek ke dalam berkas adegan. Akhirnya, akhiri berkas
dengan povend(). Secara default, raytracer akan dijalankan, dan PNG
akan dimasukkan ke dalam notebook Euler.

Fungsi objek memiliki parameter yang disebut "look", yang memerlukan
string dengan kode Povray untuk tekstur dan penyelesaian objek
tersebut. Fungsi povlook() dapat digunakan untuk menghasilkan string
ini. Ini memiliki parameter untuk warna, transparansi, Phong Shading,
dll.

Perhatikan bahwa alam semesta Povray memiliki sistem koordinat yang
berbeda. Antarmuka ini menerjemahkan semua koordinat ke sistem Povray.
Jadi Anda dapat terus berpikir dalam sistem koordinat Euler dengan z
mengarah ke atas secara vertikal, dan sumbu x, y, z dalam orientasi
tangan kanan.

Anda perlu memuat berkas povray.
\end{eulercomment}
\begin{eulerprompt}
>load povray;
\end{eulerprompt}
\begin{eulercomment}
Pastikan direktori bin Povray ada dalam path. Jika tidak, edit
variabel berikut agar mengandung path ke eksekutor Povray.
\end{eulercomment}
\begin{eulerprompt}
> defaultpovray="C:\(\backslash\)Program Files\(\backslash\)POV-Ray\(\backslash\)v3.7\(\backslash\)bin\(\backslash\)pvengine.exe"
\end{eulerprompt}
\begin{euleroutput}
  C:\(\backslash\)Program Files\(\backslash\)POV-Ray\(\backslash\)v3.7\(\backslash\)bin\(\backslash\)pvengine.exe
\end{euleroutput}
\begin{eulercomment}
Untuk kesan pertama, kita akan membuat grafik fungsi sederhana.
Perintah berikut akan menghasilkan file povray di direktori pengguna
Anda, dan menjalankan Povray untuk melacak sinar file ini.

Jika Anda menjalankan perintah berikut, GUI Povray seharusnya terbuka,
menjalankan file, dan secara otomatis menutupnya. Karena alasan
keamanan, Anda akan ditanya apakah Anda ingin mengizinkan file exe
untuk dijalankan. Anda dapat menekan "Batal" untuk menghentikan
pertanyaan lebih lanjut. Anda mungkin perlu menekan "OK" di jendela
Povray untuk mengakui dialog pemulaiannya.
\end{eulercomment}
\begin{eulerprompt}
>plot3d("x^2+y^2",zoom=2):
\end{eulerprompt}
\eulerimg{27}{images/plot 3d-056.png}
\begin{eulerprompt}
>pov3d("x^2+y^2",zoom=3);
\end{eulerprompt}
\eulerimg{28}{images/plot 3d-057.png}
\begin{eulercomment}
Kita dapat membuat fungsi tersebut menjadi transparan dan menambahkan
selesai yang lain. Kita juga dapat menambahkan garis level ke plot
fungsi tersebut.
\end{eulercomment}
\begin{eulerprompt}
>pov3d("x^2+y^3",axiscolor=red,angle=-45°,>anaglyph, ...
>  look=povlook(cyan,0.2),level=-1:0.5:1,zoom=3.8);
\end{eulerprompt}
\eulerimg{27}{images/plot 3d-058.png}
\begin{eulercomment}
Terkadang perlu untuk mencegah penskalaan fungsi, dan melakukan
penskalaan fungsi secara manual.

Kami menggambar himpunan titik-titik dalam bidang kompleks, di mana
hasil perkalian jarak ke 1 dan -1 sama dengan 1.
\end{eulercomment}
\begin{eulerprompt}
>pov3d("((x-1)^2+y^2)*((x+1)^2+y^2)/40",r=2, ...
>  angle=-120°,level=1/40,dlevel=0.005,light=[-1,1,1],height=10°,n=50, ...
>  <fscale,zoom=3.8);
\end{eulerprompt}
\eulerimg{28}{images/plot 3d-059.png}
\begin{eulercomment}
*Plotting dengan Koordinat*

Daripada menggunakan fungsi, kita dapat membuat plot dengan koordinat.
Seperti dalam plot3d, kita memerlukan tiga matriks untuk
mendefinisikan objek tersebut.

Dalam contoh ini, kita memutar sebuah fungsi sekitar sumbu z.
\end{eulercomment}
\begin{eulerprompt}
>function f(x) := x^3-x+1; ...
>x=-1:0.01:1; t=linspace(0,2pi,50)'; ...
>Z=x; X=cos(t)*f(x); Y=sin(t)*f(x); ...
>pov3d(X,Y,Z,angle=40°,look=povlook(red,0.1),height=50°,axis=0,zoom=4,light=[10,5,15]);
\end{eulerprompt}
\eulerimg{28}{images/plot 3d-060.png}
\begin{eulercomment}
Dalam contoh berikut, kita membuat grafik gelombang yang teredam. Kita
menghasilkan gelombang tersebut dengan bahasa matriks Euler.

Kita juga menunjukkan bagaimana objek tambahan dapat ditambahkan ke
dalam adegan pov3d. Untuk pembuatan objek, lihat contoh-contoh
berikutnya. Perhatikan bahwa plot3d mengubah skala plot sehingga cocok
ke dalam kubus satuan.
\end{eulercomment}
\begin{eulerprompt}
>r=linspace(0,1,80); phi=linspace(0,2pi,80)'; ...
>x=r*cos(phi); y=r*sin(phi); z=exp(-5*r)*cos(8*pi*r)/3;  ...
>pov3d(x,y,z,zoom=6,axis=0,height=30°,add=povsphere([0.5,0,0.25],0.15,povlook(red)), ...
>  w=500,h=300);
\end{eulerprompt}
\eulerimg{16}{images/plot 3d-061.png}
\begin{eulercomment}
Dengan metode shading canggih Povray, hanya sedikit titik dapat
menghasilkan permukaan yang sangat halus. Hanya di batas-batas dan
dalam bayangan trik ini mungkin menjadi jelas.

Untuk ini, kita perlu menambahkan vektor normal pada setiap titik
matriks.
\end{eulercomment}
\begin{eulerprompt}
>Z &= x^2*y^3
\end{eulerprompt}
\begin{euleroutput}
  
                                   2  3
                                  x  y
  
\end{euleroutput}
\begin{eulercomment}
Persamaan permukaannya adalah [x, y, Z]. Kami menghitung dua turunan
terhadap x dan y dari ini dan mengambil hasil perkalian silang sebagai
vektor normal.
\end{eulercomment}
\begin{eulerprompt}
>dx &= diff([x,y,Z],x); dy &= diff([x,y,Z],y);
\end{eulerprompt}
\begin{eulercomment}
Kita mendefinisikan normal sebagai hasil perkalian silang dari
turunan-turunan ini, dan mendefinisikan fungsi-fungsi koordinat.
\end{eulercomment}
\begin{eulerprompt}
>N &= crossproduct(dx,dy); NX &= N[1]; NY &= N[2]; NZ &= N[3]; N,
\end{eulerprompt}
\begin{euleroutput}
  
                                 3       2  2
                         [- 2 x y , - 3 x  y , 1]
  
\end{euleroutput}
\begin{eulercomment}
Kami hanya menggunakan 25 poin.
\end{eulercomment}
\begin{eulerprompt}
>x=-1:0.5:1; y=x';
>pov3d(x,y,Z(x,y),angle=10°, ...
>  xv=NX(x,y),yv=NY(x,y),zv=NZ(x,y),<shadow);
\end{eulerprompt}
\eulerimg{28}{images/plot 3d-062.png}
\begin{eulercomment}
Berikut adalah simpul Trefoil yang dibuat oleh A. Busser dalam Povray.
Ada versi yang ditingkatkan dari ini dalam contoh-contoh.

Lihat: Contoh\textbackslash{}Simpul Trefoil \textbar{} Simpul Trefoil

Untuk tampilan yang bagus dengan tidak terlalu banyak titik, kami
tambahkan vektor normal di sini. Kami menggunakan Maxima untuk
menghitung vektor normal untuk kita. Pertama, tiga fungsi untuk
koordinat sebagai ekspresi simbolis.
\end{eulercomment}
\begin{eulerprompt}
>X &= ((4+sin(3*y))+cos(x))*cos(2*y); ...
>Y &= ((4+sin(3*y))+cos(x))*sin(2*y); ...
>Z &= sin(x)+2*cos(3*y);
\end{eulerprompt}
\begin{eulercomment}
Kemudian dua vektor turunan terhadap x dan y.
\end{eulercomment}
\begin{eulerprompt}
>dx &= diff([X,Y,Z],x); dy &= diff([X,Y,Z],y);
\end{eulerprompt}
\begin{eulercomment}
Sekarang yang normal, yang merupakan hasil perkalian silang dari dua
turunan tersebut.
\end{eulercomment}
\begin{eulerprompt}
>dn &= crossproduct(dx,dy);
\end{eulerprompt}
\begin{eulercomment}
Sekarang kita mengevaluasi semua ini secara numerik.
\end{eulercomment}
\begin{eulerprompt}
>x:=linspace(-%pi,%pi,40); y:=linspace(-%pi,%pi,100)';
\end{eulerprompt}
\begin{eulercomment}
Vektor normal adalah hasil dari evaluasi dari ekspresi simbolik dn[i]
untuk i=1,2,3. Sintaks untuk ini adalah \&"ekspresi"(parameter). Ini
adalah alternatif dari metode dalam contoh sebelumnya, di mana kita
mendefinisikan ekspresi simbolik NX, NY, NZ terlebih dahulu.
\end{eulercomment}
\begin{eulerprompt}
>pov3d(X(x,y),Y(x,y),Z(x,y),>anaglyph,axis=0,zoom=5,w=450,h=350, ...
>  <shadow,look=povlook(blue), ...
>  xv=&"dn[1]"(x,y), yv=&"dn[2]"(x,y), zv=&"dn[3]"(x,y));
\end{eulerprompt}
\eulerimg{21}{images/plot 3d-063.png}
\begin{eulercomment}
Kita juga bisa membuat grid dalam bentuk 3D.
\end{eulercomment}
\begin{eulerprompt}
>povstart(zoom=4); ...
>x=-1:0.5:1; r=1-(x+1)^2/6; ...
>t=(0°:30°:360°)'; y=r*cos(t); z=r*sin(t); ...
>writeln(povgrid(x,y,z,d=0.02,dballs=0.05)); ...
>povend();
\end{eulerprompt}
\eulerimg{28}{images/plot 3d-064.png}
\begin{eulercomment}
Dengan povgrid(), kurva menjadi mungkin.
\end{eulercomment}
\begin{eulerprompt}
>povstart(center=[0,0,1],zoom=3.6); ...
>t=linspace(0,2,1000); r=exp(-t); ...
>x=cos(2*pi*10*t)*r; y=sin(2*pi*10*t)*r; z=t; ...
>writeln(povgrid(x,y,z,povlook(red))); ...
>writeAxis(0,2,axis=3); ...
>povend();
\end{eulerprompt}
\eulerimg{28}{images/plot 3d-065.png}
\eulerheading{Objek Povray}
\begin{eulercomment}
Di atas, kami menggunakan pov3d untuk memplot permukaan. Antarmuka
povray dalam Euler juga dapat menghasilkan objek Povray. Objek-objek
ini disimpan sebagai string dalam Euler, dan perlu ditulis ke file
Povray.

Kami memulai output dengan povstart().
\end{eulercomment}
\begin{eulerprompt}
>povstart(zoom=4);
\end{eulerprompt}
\begin{eulercomment}
Pertama, kita tentukan tiga silinder tersebut, dan simpan mereka dalam
bentuk string di Euler.

Fungsi povx() dan sebagainya hanya mengembalikan vektor [1,0,0], yang
dapat digunakan sebagai gantinya.
\end{eulercomment}
\begin{eulerprompt}
>c1=povcylinder(-povx,povx,1,povlook(red)); ...
>c2=povcylinder(-povy,povy,1,povlook(yellow)); ...
>c3=povcylinder(-povz,povz,1,povlook(blue)); ...
\end{eulerprompt}
\begin{eulercomment}
String tersebut berisi kode Povray, yang pada saat itu tidak perlu
kita pahami.
\end{eulercomment}
\begin{eulerprompt}
>c2
\end{eulerprompt}
\begin{euleroutput}
  cylinder \{ <0,0,-1>, <0,0,1>, 1
   texture \{ pigment \{ color rgb <0.941176,0.941176,0.392157> \}  \} 
   finish \{ ambient 0.2 \} 
   \}
\end{euleroutput}
\begin{eulercomment}
Seperti yang Anda lihat, kami menambahkan tekstur pada objek dalam
tiga warna berbeda.

Hal itu dilakukan dengan menggunakan povlook(), yang mengembalikan
sebuah string dengan kode Povray yang relevan. Kami dapat menggunakan
warna Euler default, atau menentukan warna sendiri. Kami juga dapat
menambahkan transparansi, atau mengubah cahaya ambien.
\end{eulercomment}
\begin{eulerprompt}
>povlook(rgb(0.1,0.2,0.3),0.1,0.5)
\end{eulerprompt}
\begin{euleroutput}
   texture \{ pigment \{ color rgbf <0.101961,0.2,0.301961,0.1> \}  \} 
   finish \{ ambient 0.5 \} 
  
\end{euleroutput}
\begin{eulercomment}
Sekarang kita mendefinisikan sebuah objek perpotongan, dan menulis
hasilnya ke dalam file.
\end{eulercomment}
\begin{eulerprompt}
>writeln(povintersection([c1,c2,c3]));
\end{eulerprompt}
\begin{eulercomment}
Perpotongan tiga silinder sulit untuk dibayangkan, jika Anda belum
pernah melihatnya sebelumnya.
\end{eulercomment}
\begin{eulerprompt}
>povend;
\end{eulerprompt}
\eulerimg{28}{images/plot 3d-066.png}
\begin{eulercomment}
Berikut adalah terjemahan dalam bahasa Indonesia sehari-hari:

Fungsi-fungsi berikut menghasilkan fraktal secara rekursif.

Fungsi pertama menunjukkan bagaimana Euler mengatasi objek Povray
sederhana. Fungsi povbox() mengembalikan sebuah string yang berisi
koordinat kotak, tekstur, dan penyelesaian.
\end{eulercomment}
\begin{eulerprompt}
>function onebox(x,y,z,d) := povbox([x,y,z],[x+d,y+d,z+d],povlook());
>function fractal (x,y,z,h,n) ...
\end{eulerprompt}
\begin{eulerudf}
   if n==1 then writeln(onebox(x,y,z,h));
   else
     h=h/3;
     fractal(x,y,z,h,n-1);
     fractal(x+2*h,y,z,h,n-1);
     fractal(x,y+2*h,z,h,n-1);
     fractal(x,y,z+2*h,h,n-1);
     fractal(x+2*h,y+2*h,z,h,n-1);
     fractal(x+2*h,y,z+2*h,h,n-1);
     fractal(x,y+2*h,z+2*h,h,n-1);
     fractal(x+2*h,y+2*h,z+2*h,h,n-1);
     fractal(x+h,y+h,z+h,h,n-1);
   endif;
  endfunction
\end{eulerudf}
\begin{eulerprompt}
>povstart(fade=10,<shadow);
>fractal(-1,-1,-1,2,4);
>povend();
\end{eulerprompt}
\eulerimg{28}{images/plot 3d-067.png}
\begin{eulercomment}
Perbedaan memungkinkan pemisahan satu objek dari objek lainnya.
Seperti persimpangan, ada bagian dari objek CSG di Povray.
\end{eulercomment}
\begin{eulerprompt}
>povstart(light=[5,-5,5],fade=10);
\end{eulerprompt}
\begin{eulercomment}
Untuk demonstrasi ini, kami mendefinisikan objek di Povray, alih-alih
menggunakan string di Euler. Definisi segera ditulis ke file.

Koordinat kotak -1 berarti [-1,-1,-1].
\end{eulercomment}
\begin{eulerprompt}
>povdefine("mycube",povbox(-1,1));
\end{eulerprompt}
\begin{eulercomment}
Kita bisa menggunakan objek ini di povobject(), yang mengembalikan
string seperti biasa.
\end{eulercomment}
\begin{eulerprompt}
>c1=povobject("mycube",povlook(red));
\end{eulerprompt}
\begin{eulercomment}
Kami membuat kubus kedua, dan memutar serta menskalakannya sedikit.
\end{eulercomment}
\begin{eulerprompt}
>c2=povobject("mycube",povlook(yellow),translate=[1,1,1], ...
>  rotate=xrotate(10°)+yrotate(10°), scale=1.2);
\end{eulerprompt}
\begin{eulercomment}
Lalu kita ambil selisih kedua benda tersebut.
\end{eulercomment}
\begin{eulerprompt}
>writeln(povdifference(c1,c2));
\end{eulerprompt}
\begin{eulercomment}
Sekarang tambahkan tiga sumbu.
\end{eulercomment}
\begin{eulerprompt}
>writeAxis(-1.2,1.2,axis=1); ...
>writeAxis(-1.2,1.2,axis=2); ...
>writeAxis(-1.2,1.2,axis=4); ...
>povend();
\end{eulerprompt}
\eulerimg{28}{images/plot 3d-068.png}
\eulerheading{Fungsi Implisit}
\begin{eulercomment}
Povray dapat memplot himpunan di mana f(x,y,z)=0, seperti parameter
implisit di plot3d. Namun hasilnya terlihat jauh lebih baik.

Sintaks untuk fungsinya sedikit berbeda. Anda tidak dapat menggunakan
keluaran ekspresi Maxima atau Euler.

\end{eulercomment}
\begin{eulerformula}
\[
((x^2+y^2-c^2)^2+(z^2-1)^2)*((y^2+z^2-c^2)^2+(x^ 2-1)^2)*((z^2+x^2-c^2)^2+(y^2-1)^2)=d
\]
\end{eulerformula}
\begin{eulerprompt}
>povstart(angle=70°,height=50°,zoom=4);
>c=0.1; d=0.1; ...
>writeln(povsurface("(pow(pow(x,2)+pow(y,2)-pow(c,2),2)+pow(pow(z,2)-1,2))*(pow(pow(y,2)+pow(z,2)-pow(c,2),2)+pow(pow(x,2)-1,2))*(pow(pow(z,2)+pow(x,2)-pow(c,2),2)+pow(pow(y,2)-1,2))-d",povlook(red))); ...
>povend();
\end{eulerprompt}
\begin{euleroutput}
  Error : Povray error!
  
  Error generated by error() command
  
  povray:
      error("Povray error!");
  Try "trace errors" to inspect local variables after errors.
  povend:
      povray(file,w,h,aspect,exit); 
\end{euleroutput}
\begin{eulerprompt}
> povstart(angle=25°,height=10°); 
>writeln(povsurface("pow(x,2)+pow(y,2)*pow(z,2)-1",povlook(blue),povbox(-2,2,"")));
>povend();
\end{eulerprompt}
\eulerimg{28}{images/plot 3d-069.png}
\begin{eulerprompt}
>povstart(angle=70°,height=50°,zoom=4);
\end{eulerprompt}
\begin{eulercomment}
Buat permukaan implisit. Perhatikan sintaksis yang berbeda dalam
ekspresi.
\end{eulercomment}
\begin{eulerprompt}
>writeln(povsurface("pow(x,2)*y-pow(y,3)-pow(z,2)",povlook(green))); ...
>writeAxes(); ...
>povend();
\end{eulerprompt}
\eulerimg{28}{images/plot 3d-070.png}
\eulerheading{Objek Jaring}
\begin{eulercomment}
Dalam contoh ini, kami menunjukkan cara membuat objek mesh, dan
menggambarnya dengan informasi tambahan.

Kita ingin memaksimalkan xy pada kondisi x+y=1 dan mendemonstrasikan
sentuhan tangensial garis datar.
\end{eulercomment}
\begin{eulerprompt}
>povstart(angle=-10°,center=[0.5,0.5,0.5],zoom=7);
\end{eulerprompt}
\begin{eulercomment}
Kita tidak dapat menyimpan objek dalam string seperti sebelumnya,
karena terlalu besar. Jadi kita mendefinisikan objek dalam file Povray
menggunakan #declare. Fungsi povtriangle() melakukan ini secara
otomatis. Ia dapat menerima vektor normal seperti pov3d().

Berikut ini definisi objek mesh, dan segera menuliskannya ke dalam
file.
\end{eulercomment}
\begin{eulerprompt}
>x=0:0.02:1; y=x'; z=x*y; vx=-y; vy=-x; vz=1;
>mesh=povtriangles(x,y,z,"",vx,vy,vz);
\end{eulerprompt}
\begin{eulercomment}
Sekarang kita mendefinisikan dua cakram, yang akan berpotongan dengan
permukaan.
\end{eulercomment}
\begin{eulerprompt}
>cl=povdisc([0.5,0.5,0],[1,1,0],2); ...
>ll=povdisc([0,0,1/4],[0,0,1],2);
\end{eulerprompt}
\begin{eulercomment}
Tulis permukaannya dikurangi kedua cakram.
\end{eulercomment}
\begin{eulerprompt}
>writeln(povdifference(mesh,povunion([cl,ll]),povlook(green)));
\end{eulerprompt}
\begin{eulercomment}
Tuliskan kedua perpotongan tersebut.
\end{eulercomment}
\begin{eulerprompt}
>writeln(povintersection([mesh,cl],povlook(red))); ...
>writeln(povintersection([mesh,ll],povlook(gray)));
\end{eulerprompt}
\begin{eulercomment}
Tulis poin maksimal.
\end{eulercomment}
\begin{eulerprompt}
>writeln(povpoint([1/2,1/2,1/4],povlook(gray),size=2*defaultpointsize));
\end{eulerprompt}
\begin{eulercomment}
Tambahkan sumbu dan selesai.
\end{eulercomment}
\begin{eulerprompt}
>writeAxes(0,1,0,1,0,1,d=0.015); ...
>povend();
\end{eulerprompt}
\eulerimg{28}{images/plot 3d-071.png}
\eulerheading{Anaglyphs in Povray}
\begin{eulercomment}
To generate an anaglyph for a red/cyan glasses, Povray must run twice
from different camera positions. It generates two Povray files and two
PNG files, which are loaded with the function loadanaglyph().

Of course, you need red/cyan glasses to view the following examples
properly.

The function pov3d() has a simple switch to generate anaglyphs.
\end{eulercomment}
\begin{eulerprompt}
>pov3d("-exp(-x^2-y^2)/2",r=2,height=45°,>anaglyph, ...
>  center=[0,0,0.5],zoom=3.5);
\end{eulerprompt}
\eulerimg{27}{images/plot 3d-072.png}
\begin{eulercomment}
If you create a scene with objects, you need to put the generation of
the scene into a function, and run it twice with different values for
the anaglyph parameter.
\end{eulercomment}
\begin{eulerprompt}
> function myscene ...
\end{eulerprompt}
\begin{eulerudf}
    s=povsphere(povc,1);
    cl=povcylinder(-povz,povz,0.5);
    clx=povobject(cl,rotate=xrotate(90°));
    cly=povobject(cl,rotate=yrotate(90°));
    c=povbox([-1,-1,0],1);
    un=povunion([cl,clx,cly,c]);
    obj=povdifference(s,un,povlook(red));
    writeln(obj);
    writeAxes();
  endfunction
\end{eulerudf}
\begin{eulercomment}
The function povanaglyph() does all this. The parameters are like in
povstart() and povend() combined.
\end{eulercomment}
\begin{eulerprompt}
>povanaglyph("myscene",zoom=4.5);
\end{eulerprompt}
\eulerimg{27}{images/plot 3d-073.png}
\eulerheading{Mendefinisikan Objek sendiri}
\begin{eulercomment}
Antarmuka povray Euler berisi banyak objek. Namun Anda tidak dibatasi
pada hal ini. Anda dapat membuat objek sendiri, yang menggabungkan
objek lain, atau merupakan objek yang benar-benar baru.

Kami mendemonstrasikan torus. Perintah Povray untuk ini adalah
"torus". Jadi kami mengembalikan string dengan perintah ini dan
parameternya. Perhatikan bahwa torus selalu berpusat pada titik asal.
\end{eulercomment}
\begin{eulerprompt}
>function povdonat (r1,r2,look="") ...
\end{eulerprompt}
\begin{eulerudf}
    return "torus \{"+r1+","+r2+look+"\}";
  endfunction
\end{eulerudf}
\begin{eulercomment}
Ini torus pertama kami.
\end{eulercomment}
\begin{eulerprompt}
>t1=povdonat(0.8,0.2)
\end{eulerprompt}
\begin{euleroutput}
  torus \{0.8,0.2\}
\end{euleroutput}
\begin{eulercomment}
Mari kita gunakan objek ini untuk membuat torus kedua, diterjemahkan
dan diputar.
\end{eulercomment}
\begin{eulerprompt}
>t2=povobject(t1,rotate=xrotate(90°),translate=[0.8,0,0])
\end{eulerprompt}
\begin{euleroutput}
  object \{ torus \{0.8,0.2\}
   rotate 90 *x 
   translate <0.8,0,0>
   \}
\end{euleroutput}
\begin{eulercomment}
Sekarang kita tempatkan objek-objek tersebut ke dalam sebuah adegan.
Untuk tampilannya kami menggunakan Phong Shading.
\end{eulercomment}
\begin{eulerprompt}
>povstart(center=[0.4,0,0],angle=0°,zoom=3.8,aspect=1.5); ...
>writeln(povobject(t1,povlook(green,phong=1))); ...
>writeln(povobject(t2,povlook(green,phong=1))); ...
\end{eulerprompt}
\begin{eulercomment}
\textgreater{}povend();

memanggil program Povray. Namun, jika terjadi kesalahan, kesalahan
tersebut tidak ditampilkan. Oleh karena itu Anda harus menggunakan

\end{eulercomment}
\begin{eulerttcomment}
 >povend(<keluar);
\end{eulerttcomment}
\begin{eulercomment}

jika ada yang tidak berhasil. Ini akan membiarkan jendela Povray
terbuka.
\end{eulercomment}
\begin{eulerprompt}
>povend(h=320,w=480);
\end{eulerprompt}
\eulerimg{18}{images/plot 3d-074.png}
\begin{eulercomment}
Berikut adalah contoh yang lebih rumit. Kami memecahkannya

\end{eulercomment}
\begin{eulerformula}
\[
Ax \le b, \quad x \ge 0, \quad c.x \to \text{Maks.}
\]
\end{eulerformula}
\begin{eulercomment}
dan menunjukkan titik-titik yang layak dan optimal dalam plot 3D.
\end{eulercomment}
\begin{eulerprompt}
>A=[10,8,4;5,6,8;6,3,2;9,5,6];
>b=[10,10,10,10]';
>c=[1,1,1];
\end{eulerprompt}
\begin{eulercomment}
Pertama, mari kita periksa, apakah contoh ini punya solusinya.
\end{eulercomment}
\begin{eulerprompt}
>x=simplex(A,b,c,>max,>check)'
\end{eulerprompt}
\begin{euleroutput}
  [0,  1,  0.5]
\end{euleroutput}
\begin{eulercomment}
Ya, sudah.

Selanjutnya kita mendefinisikan dua objek. Yang pertama adalah pesawat

\end{eulercomment}
\begin{eulerformula}
\[
a \cdot x \le b
\]
\end{eulerformula}
\begin{eulerprompt}
>function oneplane (a,b,look="") ...
\end{eulerprompt}
\begin{eulerudf}
    return povplane(a,b,look)
  endfunction
\end{eulerudf}
\begin{eulercomment}
Kemudian kita mendefinisikan perpotongan semua setengah ruang dan
sebuah kubus.
\end{eulercomment}
\begin{eulerprompt}
>function adm (A, b, r, look="") ...
\end{eulerprompt}
\begin{eulerudf}
    ol=[];
    loop 1 to rows(A); ol=ol|oneplane(A[#],b[#]); end;
    ol=ol|povbox([0,0,0],[r,r,r]);
    return povintersection(ol,look);
  endfunction
\end{eulerudf}
\begin{eulercomment}
Sekarang kita dapat merencanakan adegannya.
\end{eulercomment}
\begin{eulerprompt}
>povstart(angle=120°,center=[0.5,0.5,0.5],zoom=3.5); ...
>writeln(adm(A,b,2,povlook(green,0.4))); ...
>writeAxes(0,1.3,0,1.6,0,1.5); ...
\end{eulerprompt}
\begin{eulercomment}
Berikut ini adalah lingkaran di sekitar optimal.
\end{eulercomment}
\begin{eulerprompt}
>writeln(povintersection([povsphere(x,0.5),povplane(c,c.x')], ...
>  povlook(red,0.9)));
\end{eulerprompt}
\begin{eulercomment}
Dan kesalahan ke arah optimal.
\end{eulercomment}
\begin{eulerprompt}
>writeln(povarrow(x,c*0.5,povlook(red)));
\end{eulerprompt}
\begin{eulercomment}
Kami menambahkan teks ke layar. Teks hanyalah objek 3D. Kita perlu
menempatkan dan memutarnya sesuai dengan pandangan kita.
\end{eulercomment}
\begin{eulerprompt}
>writeln(povtext("Linear Problem",[0,0.2,1.3],size=0.05,rotate=5°)); ...
>povend();
\end{eulerprompt}
\eulerimg{28}{images/plot 3d-075.png}
\eulerheading{Contoh Lainnya}
\begin{eulercomment}
Anda dapat menemukan beberapa contoh Povray di Euler di file berikut.

See: Examples/Dandelin Spheres\\
See: Examples/Donat Math\\
See: Examples/Trefoil Knot\\
See: Examples/Optimization by Affine Scaling

\begin{eulercomment}
\eulerheading{Contoh Soal}
\begin{eulercomment}
1. Buatlah grafik fungsi dari :\\
\end{eulercomment}
\begin{eulerformula}
\[
f(x,y) = -4x^3 y^2
\]
\end{eulerformula}
\begin{eulerprompt}
>aspect(2); plot3d("-4*x^3*y^2"):
\end{eulerprompt}
\eulerimg{13}{images/plot 3d-076.png}
\begin{eulercomment}
2. Buatlah grafik dengan grid dari fungsi :\\
\end{eulercomment}
\begin{eulerformula}
\[
f(x,y) = x^2+y^2
\]
\end{eulerformula}
\begin{eulerprompt}
>aspect(1.5); plot3d("x^2+y^2",>transparent,grid=10,wirecolor=red):
\end{eulerprompt}
\eulerimg{17}{images/plot 3d-077.png}
\begin{eulerprompt}
>    
\end{eulerprompt}
\begin{eulercomment}
3. Buatlah grafik seperti pada nomor 2 namun menggunakan red/cyan
glasses
\end{eulercomment}
\begin{eulerprompt}
>aspect(1.5); plot3d("x^2+y^2",>wire,>anaglyph,title="Use Red/Cyan Glasses!",n=10):
\end{eulerprompt}
\eulerimg{17}{images/plot 3d-078.png}
\begin{eulercomment}
4. Buatlah bentuk geometri silinder menggunakan plot3d
\end{eulercomment}
\begin{eulerprompt}
>plot3d("x","cos(y)","sin(y)",xmin=-2,xmax=2,ymin=0,ymax=2pi, ...
>scale=[1.5,1.5,1.5],n=100,grid=10,fillcolor=[yellow,orange]):
\end{eulerprompt}
\eulerimg{17}{images/plot 3d-079.png}
\begin{eulercomment}
5. Buatlah grafik dari fungsi :\\
\end{eulercomment}
\begin{eulerformula}
\[
f(x,y,z) = x^2+y3+sin(z)^2-1
\]
\end{eulerformula}
\begin{eulerprompt}
>plot3d("x^2+y^3+sin(z)^2-1",r=pi,implicit=4,zoom=3):
\end{eulerprompt}
\eulerimg{17}{images/plot 3d-080.png}

